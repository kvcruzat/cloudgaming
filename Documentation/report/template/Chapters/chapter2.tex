\chapter{Background Research}
\label{chapter2}

\section{Problem Overview}
Compute power of games can be offloaded to a server of much powerful computers and then streamed to a client with lower specification hardware such as a laptop or a mobile device. With this comes risks and shortcomings that need to be factored. One of these problems are the latency of the game. Latency can be huge factor in the gameplay such as high-paced games like first-person shooters (FPS) or fighting games. The interaction delay in pressing a button on the gamepad to seeing the action performed on the screen needs to be kept to a minimum. This idea of interaction delay tolerance being different from genre to genre of games is discussed by Shea et al. \cite{shea2013cloud}. As stated above, a player of FPS games can only tolerate the least which is around 100ms whereas role-playing game (RPG) gamers and turn-based game gamers can tolerate up to around 500 ms.
\newline
\par
Another problem that is directly linked to delay in the system is the effect of packet loss. As stated in the \textit{Eight Fallacies of Distributed Computing} \cite{deutsch1994eight}, it should be assumed that latency is never zero as mentioned above as well as network is not always reliable. This means that packet loss can occur which in terms of cloud computing can mean the degradation of image quality. In the investigation conducted by Jarschel et al. \cite{jarschel2011evaluation} in which they surveyed average consumers about the importance of packet loss and delay. Generally the quality of the video streamed to the clients plays an important role as the participants were open to using such as a service if provided in good quality. Delay is an important factor when it comes to gaming so this project focuses on tackling this problem. The background of cloud gaming and cloud computing was explored to better understand the nature of latency and its source in distributed systems.

\section{Cloud Computing}
According to the National Institute of Standards and Technology \cite{mell2011nist}, cloud computing is a means of providing on demand access to computing resources over the network.  This should be executed with minimal management effort or service provider interaction. With the shared computing resources, cloud computing aims to process larger data and solve large scale computation. A remotely hosted data centre that provides services that is otherwise unachievable with local machines with lower compute resources has been a popular innovation in recent years to commercial companies.
\newline
\par
This cloud computing model can be separated in to three different service models: Software as a Service (SaaS), Platform as a Service (PaaS), Infrastructure as a Service (IaaS) \cite{jadeja2012cloud}. 'SaaS' model attempts to eliminate the need to install and run the application on the user's system. An example of this is the Microsoft Office 365 package which provides productivity software through the web browser. 'PaaS' model provides the consumer a computing platform using the cloud infrastructure to allow running and building their own applications. The consumer does not need to manage the "underlying cloud infrastructure including networks, servers, operating systems or storage \cite{mell2011nist}". 'IaaS' model provides the capability to control the processing, storage and networks to an extent. The consumers will have to install the OS images and related application software on the cloud infrastructure \cite{cloudservices}.
\newline
\par
A disadvantage stated by Grossman \cite{grossman2009case} is that since cloud services are often remote, it can suffer from latency and bandwidth related issues. Data centres can be physically located anywhere in the world, so the number of router hops from the client to the server may attribute to the latency. As well as distance, delays are also introduced by network hardware and error correction on data packets.

\section{Cloud Gaming}
Cloud gaming is new technology that can be seen as an alternative by having the games run remotely on a server and then streamed to the user. Performing computations remotely as with streaming games remotely is believed to gain traction in the future in the same way how streaming videos and audio have become ubiquitous through services such as Netflix and Spotify. NVIDIA's GRID Cloud Gaming advancements have shown that this is becoming the case. As stated by Mariano in \textit{Is cloud gaming the future of the gaming industry} \cite{mariano2015cloud}, cloud gaming is increasingly becoming an attractive option for consumers as higher end games can then work on simpler, cheaper clients as well as with devices that they may already own also known as thin clients through the use of powerful server GPUs.
\newline
\par
These thin clients are responsible for displaying the game frames rendered on the cloud server side in the form of video frames. Also, it has to collect and process the game control inputs from the user and send these to cloud to be registered as inputs on the game engine. According to Shea et al \cite{shea2013cloud}, cloud gaming would be of great benefit to the game industry as it would open the user base to the thin clients. For example the recommended specifications to run the 2015 Game of the Year title Witcher 3\cite{goty2015} would require a system that has \cite{witcher}:
\begin{itemize}
 \item CPU: Intel Core i7 3770 3.4 GHz / AMD FX-8350 4 GHz
 \item GPU: GeForce GTX 770 / AMD Radeon R9 290
 \item RAM: 8GB
\end{itemize}
A system that has these components would cost around \textlira400 and this does not include peripherals such as keyboard, mouse and monitor. The latest tablets and laptops can barely if at all come close to meeting the recommended specifications to run the game natively. Furthermore, games will have to deal with running on different hardware architectures and operating systems. Cloud gaming will make it easier for developers as they will not have to deal with platform compatibility and per platform tuning. This means that users of cloud gaming services will be capable of running games with the same performance and quality as other users. Different game consoles and computers and their diversity of architectures and hardware lead to fragmentation in player experiences with lower end machines providing worse gameplay experiences. Cloud gaming would eliminate this problem in the gaming industry.
\newline
\par
In the paper \textit{Cloud Gaming: A Green Solution to Massive Multiplayer Online Games} \cite{chuah2014cloud} it mentions that NVIDIA has introduced SHIELD which is a mobile gaming device that can be connected to a desktop PC with a compatible NVIDIA GPU and stream gameplay to the device via 802.11n WiFi. Another feature GeForce Now, has the ability to connect to one of NVIDIA's data centres around the world to play games from their vast library of stream-ready games. One of the benefits of this service is the convenience of not having to wait for the download and installation of the game as you simply pick a game and instantly start playing this is because the games are already installed and optimised in their servers. The games will not be stored on the client's device so storage limitations that come with natively playing games will not be a problem. The service also boasts gameplay performance of up to 1080p resolution at 60 frames-per-second which are deemed by gamers to be the target performance for a smooth, sharp gameplay experience.
\newline
\par
A simple cloud gaming architecture consists of several procedures which adds on top of the game engine process as shown in Figure \ref{fig:arch}. The client would send user inputs through some form of controller like a keyboard, mouse, or gamepad to the thin client. The game client then encodes these commands so it can easily be sent to the game server via network. When the server receives the user inputs, it simulates them so the game that is running recognises them. The game renders the frames produced as a result of the inputs and are captured and encoded to video frames. This is done so it can be easily streamed to the client through the use of Real Time Streaming Protocol (RTSP) which is a network control protocol that manages delivery of data with real time properties \cite{rtsp}. The thin client uses RTSP to receive the video frames and decodes them to be displayed on a video player. The user then sees the results of the button presses sent from the game running on the server side.

\begin{figure}[h!]
 \includegraphics[width=\linewidth]{images/arch.png}
 \caption{Cloud gaming system architecture \cite{cloudarch}}
 \label{fig:arch}
\end{figure}

Due to the many different processes involved in the architecture of a cloud gaming system, managing latency has become a problem. A traditional gaming system already experiences latency and as shown in Figure \ref{fig:latency}, this arises from the game pipeline and display lag. The game pipeline latency is the amount of time it takes for the game to compute and render a frame and the display lag is the time it takes to display the frames on the screen. Display lag can be caused by the display's scaler since current displays have a fixed resolution and expensive image processing such as dynamic contrast and motion interpolation \cite{displaylag}. Cloud gaming introduces latency from capturing/encoding game frames to video frames on the server side, network latency with the transmission of data to both sides and the decoding of the video for playback on the client side.
\clearpage
\begin{figure}[h!]
 \centering
 \includegraphics[width=\linewidth]{images/latency.png}
 \caption{Latency in cloud gaming \cite{cloudlatency}}
 \label{fig:latency}
\end{figure}

The aim of this project is to reduce the network latency which is the element that introduces the most latency in cloud gaming.

\section{Latency Mitigation}
One method of latency mitigation in the network is to simply move the server closer to the clients. This means that traffic will have to travel less distance therefore latency will decrease. Unfortunately, this solution is not feasible since building and maintaining data centres are expensive. Video streaming services such as Netflix and YouTube use buffering which loads video data before playing the video in order for continuous playback. This cannot be done for live cloud gaming so other solutions to improve experiences of latency sensitive games was explored.

\subsection{Speculative Execution: Outatime}
Microsoft's Outatime which uses \textit{speculation to enable low-latency continuous interaction for mobile cloud gaming} \cite{lee2015outatime} is a form of latency mitigation. This method acknowledges that there will be latency and attempts to reduce its effects since studies have shown that players are sensitive to latency from 75-100ms, decreasing accuracy in shooting games and decreasing their scores by almost up to 50\% \cite{beigbeder2004effects}.
\newline 
\par
Outatime basically predicts multiple possible frame outputs that may appear in the future of the game's render scene on the client side. It has to predict what frames may be needed at least a full end host round trip time (RTT) ahead of time the client actually produces game input controls. So unlike standard cloud gaming where clients receive a response after more than one RTT, Outatime delivers a response immediately since the possible frames are all ready to be displayed.
\newline
\par
Outatime was extensively tested by having people play a shooting game without knowing if they were using a traditional cloud gaming system or Outatime. The tests were conducted multiple times at various network delays and players were asked qualitatively their opinion on the playability of the game. The results show that Outatime was favoured by the players in all cases even with experienced players.
\newline
\par
This method to mitigate latency by masking it through ahead of time render speculation is a good idea to explore and improve upon, but was not chosen for this project. This is due to Outatime being a closed source software with no available source code. Even though the basic mechanisms of how Outatime works is outlined, developing a working solution that uses them will not be able to in completed for this project due to the lack of technical skills and experience as well as time the time to acquire them.

\subsection{Software-Defined Networking}
Another form of latency mitigation that is being explored is software-defined networking. As stated by Kirkpatrick \cite{kirkpatrick2013software}, software-defined networking (SDN) is a new networking architecture that allows programmers to quickly reconfigure and define network usage. Whilst significant advances have been in other areas of technology, networking has not been able to evolve in the same pace.
\newline
\par
Similar to mobile phones shifting to the world of smartphones with the help of APIs (application program interface), in an SDN environment, applications can communicate with network switches through an API. The API can be used to quickly reconfigure the resources of the network to accommodate the needs of the applications being executed. This main benefit of using SDN is also discussed in \textit{Improving network management with software defined networking} \cite{kim2013improving}. Kim et al mentions that network operators will not need to configure all the network devices individually to make network behaviour changes, but instead make network-wide traffic forwarding decisions. The SDN controller is used for this and would have global knowledge of the state of the network.
\newline
\par
SDN consists of two planes, the data and control plane. The data plane also known as the forwarding plane is the part of the network that carries user traffic by forwarding them to the next hop along the path to its destination in accordance to the logic in the control plane \cite{dataplane}. The control plane has command where the traffic is sent by creating the routing tables and is also responsible for managing connections between switches, handling errors and exceptions \cite{controlplane}. This separation allows the control plane to be directly programmable and the network elements in the data plane to be abstracted for networking services.
\newline
\par
The controller is the core component for the software-defined networking. It is an application that manages the flow control of data between nodes. In order for SDN applications to communicate with the network switches it has to go through the controller which can be accessed by applications with REST API (a way to communicate to between machines through HTTP requests). This allows network administrators to manage how the data plane handles network traffic and its routing.

\section{Related Works}
There are a few researches that have been conducted that is related to latency mitigation in cloud gaming systems using software-defined networking. Amiri et al. \cite{amiri2015sdn1} explored using SDN to reduce latency by using SDN to share the load in a network. The shortest path is found in the network and the path cost is computed by pinging the machines at each node to discover the end-to-end delay. Incoming packets are then split among the paths depending on the weights so the load is evenly shared in the network.
\newline
\par
Another research that is related is also by Amiri et al. \cite{amiri2015sdn2} and aims to reduce network latency by considering the type of game being requested and using this to optimally assign a game server using this property. Different types of games have different network delay sensitivities such as a shooting game would have a higher sensitivity than a turn-based game. With this taken in to account it uses a weighted priority system when using SDN to find game servers that have paths that allow for this network delay.
\newline
\par
This project used this idea of load sharing, but measures the current load in each path and forwards traffic to the path with the lowest load dynamically. Also, multiple test cases that reflect real world scenarios of cloud gaming system usage are used to better understand the effectiveness of software-defined networking. The design of the load balancing application is discussed in section 3.4