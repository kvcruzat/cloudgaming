\chapter{Background Research}
\label{chapter2}

\section{Problem Overview}
Compute power of games can be offloaded to a server of much powerful computers and then streamed to a client with lower specification hardware such as a laptop or a mobile device. With this comes risks and shortcomings that need to be factored. One of these problems are the latency of the game. Latency can be huge factor in the gameplay such as high-paced games like first-person shooters or fighting games. The delay in pressing a button on the gamepad to seeing the action performed on the screen needs to be kept to a minimum. This idea of interaction delay tolerance being different from genre to genre of games is discussed by Shea et al. \cite{shea2013cloud}. As stated above, a player of FPS games can only tolerate the least which is around 100ms whereas Role playing game (RPG) gamers can tolerate around 500 ms.
\par
Another problem that is directly linked to delay in the system is the effect of packet loss. As stated in the \textit{Eight Fallacies of Distributed Computing} \cite{deutsch1994eight}, it should be assumed that latency is never zero as mentioned above as well as network is not always reliable. This means that packet loss can occur which in terms of cloud computing can mean the degradation of image quality. In the investigation conducted by Jarschel et al. \cite{jarschel2011evaluation} in which they surveyed average consumers about the importance of packet loss and delay. Generally the quality of the video streamed to the clients plays an important role as the participants were open to using such as a service if provided in good quality.

\section{Cloud Computing}
\lipsum[1-1]

\section{Cloud Gaming}
Cloud gaming is new technology that can be seen as an alternative by having the games run remotely on a server and then streamed to the user. Performing computations remotely as with streaming games remotely is believed to gain traction in the future in the same way how streaming videos and audio have become ubiquitous through services such as Netflix and Spotify. NVIDIA's GRID Cloud Gaming advancements have shown that this is becoming the case. As stated by Mariano in \textit{Is cloud gaming the future of the gaming industry} \cite{mariano2015cloud}, cloud gaming is increasingly becoming an attractive option for consumers as higher end games can then work on simpler, cheaper clients as well as with devices that they may already own.
\parls
In the paper \textit{Cloud Gaming: A Green Solution to Massive Multiplayer Online Games} \cite{chuah2014cloud} it mentions that NVIDIA has introduced SHIELD which is a mobile gaming device that can be connected a desktop PC with a compatible NVIDIA GPU and stream gameplay to the device via 802.11n WiFi. Another feature is the ability to connect to one of NVIDIA's data centres to play games from their selection of stream-ready games. One of the benefits of this service is the convenience of not having to wait for the download and installation of the game as you simply pick a game and instantly start playing. The service also boasts gameplay performance of up to 1080p at 60fps.

\section{Latency Mitigation}
\lipsum[1-1]

\section{Related Works}
\lipsum[1-1]