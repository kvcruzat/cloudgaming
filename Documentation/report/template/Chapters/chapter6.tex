\chapter{Conclusion}
\label{chapter6}

\section{Conclusion}
For this project, the problem that was undertaken was to explore the feasibility and the effects of using software-defined networking in a cloud gaming system. In order to do this a cloud gaming system and a game was implemented to better understand latency and how it arises in cloud gaming. Background research and literature review was conducted so the problem can be specified and narrowed down to a size where the project can be undertaken within the time constraints. Also, research was needed to be done on the feasibility of the solution to see if it could be done with current technical and programming skills.
\newline
\par
A simple game with flight simulator controls was designed and implemented. The game also modelled a real time procedurally generated tree with lighting and shadows which uses expensive compute resources so moving it to a more powerful data centre would be beneficial. A cloud gaming system was attempted to be implemented but was not fully accomplished. A solution was complete up to the point of sending user inputs and simulating them to the game on the server side. The video streaming of the game frames to the client was not finished which meant the amount of traffic produced by it could not be recorded. This was supposed to be used with the software-defined network tests to generate traffic of the same throughput.
\newline
\par
For the networking portion of the solution to the problem, a virtual network was created and used to simulate a data centre network that could have been used in a cloud gaming system along with latency and bandwidth limits in the network links. Software-defined networking is then used by deploying a load balancer application on to the network with the help of a SDN controller. The load balancer finds the shortest paths between two nodes in the network and computes the usage of the links to find the least used route. This route is then pushed to the SDN controller so any traffic between the two nodes use this path. Multiple instances of this load balancer can be running so every flow can be load balanced also it will update periodically so the load can be balanced even when new flows are introduced which made it dynamic.
\newline
\par
With the results from the different test scenarios, it has shown that deploying load balancer using software-defined networking helps with reducing latency in a network. Under different loads and scenarios, load balancing generally improved the ping times between player hosts and game servers. It was concluded that in the tests where ping times did not improve or improved by a small amount, it was due to trying to force throughput higher than the bandwidth limit in the link. A limitation with the tests is that it was conducted with simulated networks, traffic and parameters which cannot fully replicate a real network so results may vary in a real case. With this aside, the project has shown that software-defined networking in terms of dynamic load balancing improves network latency in a cloud gaming system so therefore a player's gameplay experience would be improved with this reduction of lag.

\section{Future Work}
There are many areas in the project which can be improved on or how it could be used to carry out further research. This section outlines this and how it would give better, reliable results and insight to the problem.
\newline
\par
One of the improvements is to complete the cloud gaming system by implementing the video streaming aspect. This is to get the rendered frames on the server side encoded in to video streaming format and sent to the client where the client program displays it on the window.The planned implementation ideas were outlined in section 4.2 and if this was completed the video traffic produced can be recorded using a packet analyser like Wireshark which would give more accurate simulated traffic in the test cases. Also, interaction delay which is the amount of time for a button to be pressed and the corresponding action displayed on the screen can then be measured.
\newline
\par
Another improvement that could be made is to deploy the solution on a real world data centre which has support for software-define networking. This means implementing a fully working cloud gaming system where it would create a virtual machine and run a separate game instance for each connected client as mentioned in the design chapter. The virtual machines would be dynamically customised to the parameters set by the client in order meet the performance requirements set. Also, a fully functioning software-defined networking solution to be deployed on the data centre where it would automatically run a load balancer instance for each new client and have it run in the background whilst constantly updating for new load balanced routes. This will give the most accurate test results since it is deployed on a real physical network with no simulated data.
\newline
\par
An improvement that would be interesting to implement once a full cloud gaming system is complete is to develop a game that is capable of multiple players running the same game instance so essentially a multiplayer game. With multiple players communicating with the same game engine and logic, seeing how deploying this on a cloud gaming system affects the latency and how effective load balancing is with this case will give results that would be helpful for future cloud multiplayer gaming implementations.
\newline
\par
Further research can be done using the outcomes achieved in this project. The load balancing method can be improved and optimsed to give better results and update more dynamically as new load are only taken in to account every minute. Other load balancing methods can also be designed and developed and this project can be used to compare the effectiveness. Software-defined networking is not just limited to load balancing to reduce latency, other means of congestion control or latency mitigation can be better solutions and this project will serve as a good entry level to explore other solutions.

\section{Personal Reflection}
In this project, there were aspects that proved to be more difficult than initially expected and this section will highlight them and methods to avoid or reduce their effects if the project or a similar project was to be undertaken again.
\newline
\par
One of the areas that I deemed to be the most difficult is in the beginning stages where the problem the project is trying to solve needs to be specified. At first, I was under the impression that the project would just consist of developing a game and cloud gaming system that would offload the computation on the server side. I found it difficult to find a problem case that I could handle with my technical skills under the imposed time constraint. Background research proved to aid finding a problem and a feasible solution to it. The problem was not enough research was done early on and not having specific problem and solution to work with brought up other problems.
\newline
\par
A problem that came up is that many solutions that was attempted in the implementation stage came to a dead end. These road blocks were met and more background research had to be done to find alternative solutions. The main example of this is with the use of the School of Computing cloud testbed. From early on in the project I was already preparing the environment and virtual machines that will be used for deploying the cloud game system and game. There was problems in connecting with the cloud remotely off campus and problems with trying to get a working OpenGL environment. Another problem was when trying to get the software-defined networking controller and virtual switches working, but was not successful upon finding out Open vSwitch not being enabled. So a virtual network was the alternative solution at the end, but a lot of time was consumed due to the lack of a solid plan and research. If knowledge of SDN being used along with its requirements was known earlier on then there would have been more time on other aspects of the project.
\newline
\par
Another challenging aspect of the project was learning software-defined networking. With the distributed systems module and past internship with HPC in the university, I already had prior knowledge of networking and data centre architecture, but software-defined networking was a new paradigm that I had trouble understanding. I tried working with SDN controllers and virtual switches with not enough background reading and lead to a slower learning rate.
\newline
\par
On the other hand, an area that went smoothly is during the developing stage of the game. This is because of the knowledge and experience from the computer graphics module that was recently completed. I managed to complete the development of the game within the allotted time frame given in the project schedule. Previous development in OpenGL and Qt was very useful in producing the game which gave more time to be spent on the rest of the project
\newline
\par
In conclusion, the project was a challenging experience with its many setbacks, but rewarding due to the knowledge gained from it. An important factor to take away from this project is to efficiently conduct background research and plan early before getting too deep with the development. Without proper understanding of the problem and preparation for the development stage, many routes in developing can lead to dead ends and be time consuming. What I felt that was done effectively was adapting to these problems and finding alternative solutions. This helped in producing a solution that answers the question of the project therefore the objective was met.